\documentclass[12pt]{article}
\usepackage{polyglossia}
\usepackage{fontspec}
\usepackage[export]{adjustbox}
\usepackage{amsfonts, amsmath, amssymb}
\usepackage{bm}
% \usepackage{breqn}
\usepackage{derivative}
\usepackage{empheq}
\usepackage[inline]{enumitem}
\usepackage{fancyhdr}
\usepackage{geometry}
\usepackage{graphicx}
\usepackage{kantlipsum}
\usepackage{lastpage}
\usepackage{linebreaker} % line-breaker algorithm in LuaLaTeX
\usepackage{lualatex-math} % Fixes for mathematics
\usepackage{mathtools}
\usepackage{microtype}
\usepackage{mismath}
\usepackage{nicefrac}
\usepackage[hyperref]{ntheorem}
% \usepackage{phfqit} % Bra-ket notation
% \usepackage{physics}
\usepackage[final]{showlabels}
\usepackage{simples-matrices}
\usepackage{siunitx}
\usepackage{soul}
\usepackage{titling}
\usepackage{tabularray}
\usepackage{tasks}
\usepackage{tensor}
\usepackage{xcolor}
\usepackage{xfrac}
\usepackage{xparse}
\usepackage{xspace}
\usepackage{hyperref}
% Polyglossia config
\setdefaultlanguage[variant=mexican]{spanish}
\setotherlanguage{english}
\usepackage{cleveref}

%%%%%%%%%%%%%%%%%%%%%%%%%%%%%%%%%%%%%%%%%%%%%%%%%%%%%%%%%%
%%%%%%%%%%%%%%%%%%%%%%%%%%%%%%%%%%%%%%%%%%%%%%%%%%%%%%%%%%
%%%%%%%%%%%%%%%%%%Configuraciones extras%%%%%%%%%%%%%%%%%%
\definecolor{base3}{RGB}{253, 246, 227}%
\definecolor{pinkwave}{RGB}{255, 0, 128}%
\definecolor{customBlue}{RGB}{11, 61, 98}%
\pagecolor{base3}
\graphicspath{{img/}}
\setlength{\parindent}{2em} % Sangría
\setlength{\parskip}{0.5em} % Espacio entre párrafos
\linespread{1.1} % line spacing
\setlength{\jot}{10pt} % Space between lines in multiline eqs
\crefname{equation}{ec.}{ecs.} % Equation's cross-reference name

% Line-breaker config
\linebreakersetup {
    maxtolerance = 90,
    maxemergencystretch = 1em,
    maxcycles = 4
}

%%%%%%%%%%%%%%%%%%Theorem environments%%%%%%%%%%%%%%%%%%
% Configuración de ambiente para problema
\theoremstyle{break}
\theoremheaderfont{\Large\normalfont\bfseries}
\theorembodyfont{\normalfont}
\theoremseparator{\bigskip} % Spacing between header and body
\theorempreskip{1.5em}
\theorempostskip{\topsep\bigskip}
\theorempostwork{
    \color{customBlue} \hrule width \hsize height 2pt \kern 1mm \hrule width \hsize
    }
\newtheorem{exercise}{Problema}
\counterwithin{equation}{exercise}

% Configuración de ambiente para solución
\theoremstyle{nonumberbreak}
\theoremheaderfont{\Large\normalfont\bfseries}
\theorembodyfont{\normalfont}
\theoremseparator{\medskip}
\theorempreskip{1em}
\theorempostskip{\topsep\medskip}
\newtheorem{solution}{Solución}

%%%%%%%%%%%%%%%%%%%%%%%%%%%%%%%%%%%%%%%%%%%%%%%%%%%%%%%%


% Configuración del paquete hyperref
\hypersetup{
    colorlinks = true,%
    linkcolor={[rgb]{0,0.2,0.6}},%
    citecolor={[rgb]{0,0.6,0.2}},%
    filecolor={[rgb]{0.8,0,0.8}},%
    urlcolor={[rgb]{0.8,0,0.8}},%
    runcolor={[rgb]{0.8,0,0.8}},% 
    menucolor={[rgb]{0,0.2,0.6}},%
    linkbordercolor={[rgb]{0,0.2,0.6}},%
    citebordercolor={[rgb]{0,0.6,0.2}},%
    filebordercolor={[rgb]{0.8,0,0.8}},%
    urlbordercolor={[rgb]{0.8,0,0.8}},%
    runbordercolor={[rgb]{0.8,0,0.8}},%
    menubordercolor={[rgb]{0,0.2,0.6}},% 
    pdftitle={Tarea 3},%
    pdfauthor={López Merino Marcos},%
    pdfsubject={Relatividad},%
    pdfkeywords={Facultad de Ciencias, UNAM, Relatividad},%
    unicode = true%
}

%%%%%%%%%%%%%%%%%%siunitx configuration%%%%%%%%%%%%%%%%%%
% Configuración del paquete siunitx
\sisetup{
	output-decimal-marker = {.}, 
	per-mode = symbol-or-fraction,
	separate-uncertainty = false,
	exponent-product = \cross,
    % inter-unit-product = \ensuremath{{}\vdot{}}
}

% Declaring new units
\DeclareSIUnit\kilogram{\kilo\gram}

%%%%%%%%%%%%%%%%%%%%%%%%%%%%%%%%%%%%%%%%%%%%%%%%%%%%%%%%

% Geometría del documento
\geometry{
    letterpaper,
    top = 0.6in,
    bottom = 0.8in,
    left = 0.6in,
    right = 0.6in,
    footskip = 38pt
}

%%%%%%%%%%%%%%%%%%Nuevos comandos%%%%%%%%%%%%%%%%%%
\newcommand*{\group}{8231}
\newcommand*{\classname}{Relatividad}
\newcommand*{\homeworknumber}{Tarea 4}
\newcommand*{\name}{Marcos López Merino}

% unit vector i
\newcommand*{\uveci}{{\bm{\hat{\textnormal{\bfseries\i}}}}}
% unit vector j
\newcommand*{\uvecj}{{\bm{\hat{\textnormal{\bfseries\j}}}}}
% unit vector
\DeclareRobustCommand{\uvec}[1]{{%
  \ifcsname uvec#1\endcsname
     \csname uvec#1\endcsname
   \else
    \bm{\hat{\mathbf{#1}}}%
   \fi
}}% 
\newcommand{\idest}{\emph{i.e.},\xspace} % id est
% Espacio vectorial, e.g., ℝ, ℂ, ℕ, etc.
\NewDocumentCommand{\vecspace}{m o}{%
  \IfValueTF{#2}{%
    \mathbb{#1}^{#2}%
  }{%
    \mathbb{#1}%
  }%
}

\newcommand*{\observer}{\mathcal{O}}
\newcommand*{\primeobserver}{\mathcal{O}^{\prime}}
\newcommand*{\biprimeobserver}{\mathcal{O}^{\prime\prime}}
\newcommand*{\triprimeobserver}{\mathcal{O}^{\prime\prime\prime}}
\newcommand*{\inlinesol}{\vspace*{10pt}\textbf{Solución}\vspace*{10pt}}
% \newcommand*{\change}[1]{\Delta{#1}\xspace}

\NewDocumentCommand{\change}{ s o m }{%
    \IfBooleanTF{#1}{%
        \Delta {#3^{\prime}}
    }{%
        \Delta {#3}
    }
}
\NewDocumentCommand{\interval}{ s o m }{%
    \IfBooleanTF{#1}{%
        (\Delta {#3^{\prime}})^{2}
    }{%
        (\Delta {#3})^{2}
    }
}
% \newcommand{\crefrangeconjunction}{--}
% \newcommand{\crefpairconjunction}{\xspace y\xspace}

% Defining a variant of Aboxed command from mathtools
\makeatletter
\newcommand*\Acolorboxed[2][pinkwave]{%
   \let\bgroup{\romannumeral-`}%
   \@Acolorboxed{#1}#2&&\ENDDNE
}
\def\@Acolorboxed#1#2&#3&#4\ENDDNE{%
  \ifnum0=`{}\fi
  \setbox\z@\hbox{$\displaystyle#2{}\m@th$\kern\fboxsep \kern\fboxrule}%
  \edef\@tempa{\kern\wd\z@ & \kern-\the\wd\z@ \fboxsep\the\fboxsep \fboxrule\the\fboxrule}%
  \@tempa
  \fcolorbox{#1}{base3}{\m@th$\displaystyle#2#3$}%
} 
\makeatother

%%%%%%%%%%%%%%%%%%Portada y configuración%%%%%%%%%%%%%%%%%%
% Configuración de portada
\setlength{\droptitle}{-60pt} % raise the title

% Portada
\title{
    \textbf{\homeworknumber}\\
    \normalsize\vspace{0.1in}\small{\textbf{Entrega}:~\today}
    \vspace{-1.5in}
}
\author{}
\date{}

%%%%%%%%%%%%%%%%%%Header and footer%%%%%%%%%%%%%%%%%%
\setlength{\headheight}{15.2pt}
\pagestyle{fancy}
\lhead{Grupo \group, Sem. 2023-1}
\chead{\classname}
\rhead{\name}
\lfoot{\includegraphics[scale = 0.2, valign = c]{LogoFCUNAMcolor.pdf}}
% \lfoot{\includegraphics[scale = 0.1, valign = c]{example-image}}
\cfoot{\homeworknumber}
\rfoot{Pág. \thepage \hspace{1pt} de \pageref{LastPage}}

\renewcommand{\headrulewidth}{0.5pt}
\renewcommand{\footrulewidth}{0.5pt}
%%%%%%%%%%%%%%%%%%%%%%%%%%%%%%%%%%%%%%%%%%%%%%%%%%%%%%%%%%
%%%%%%%%%%%%%%%%%%%%%%%%%%%%%%%%%%%%%%%%%%%%%%%%%%%%%%%%%%

\begin{document}
    \maketitle
    \thispagestyle{fancy}
    
    \begin{exercise}
        \begin{enumerate}[label = \alph*)]
            \item Considera un observador inercial dentro de un elevador estático sobre una superficie carga infinita, de forma tal que hay un campo eléctrico constante \(\vect{E} = -E \uvec{z}\) en todo el espacio. No hay ningún otro tipo de campo gravitacional en el espacio. El observador deja caer dos partículas desde la misma altura: la primera tiene carga eléctrica \(q\) y masa \(m\), mientras que la segunda tiene carga \(q\) y masa \(M\), tal que \(M > m\). ¿Cuál de las dos partículas llega primero al suelo del elevador según el observador dentro? La interacción eléctrica y gravitacional entre ambas partículas es despreciable.
            \item Considera un observador inercial dentro de un elevador estático sobre una superficie infinita, de forma tal que ésta genera un campo gravitacional constante \(\vect{G} = -g\uvec{z}\) en todo el espacio. No hay ningún tipo de campo electromagnético en el espacio. El observador deja caer dos partículas desde la misma altura: la primera tienen carga eléctrica \(q\) y masa \(m\), mientras que la segunda tiene carga \(q\) y masa \(M\), tal que \(M > m\). ¿Cuál de las dos partículas llega primero al suelo del elevador según el observador dentro? La interacción eléctrica y gravitacional entre ambas partículas es despreciable.
            \item Considera un observador no inercial dentro de un elevador que se mueve en el espacio con aceleración constante \(\vect{a} = a\uvec{z}\) respecto a otro sistema inercial externo al elevador. No hay ningún campo gravitacional ni electromagnético en el espacio. El observador no inercial en el elevador deja caer dos partículas desde la misma altura: la primera tiene carga eléctrica \(q\) y masa \(m\), mientras que la segunda tiene carga \(q\) y masa \(M\), tal que \(M > m\). ¿Cuál de las dos partículas llega primero al suelo del elevador según el observador dentro? La interacción eléctrica y gravitacional entre ambas partículas es despreciable.
        \end{enumerate}
    \end{exercise}

    \pagebreak
    \begin{exercise}
        \begin{enumerate}[label = \alph*)]
            \item De acuerdo a la Relatividad General, ¿qué tipo de trayectoria en el espacio-tiempo sigue un cuerpo libre de fuerzas? En palabras simples, explica la definición de ese tipo de trayectoria y da un ejemplo sencillo. No necesitas mayor formalidad matemática que la que seguimos en clase.
            
            \inlinesol

            El tipo de trayectoria que sigue un cuerpo libre de fuerzas en el espacio-tiempo es una \setulcolor{pinkwave}\ul{\textbf{parábola}}.\par

            Este tipo de trayectoria es conocida como \textbf{geodésica}, que es la trayectoria más recta posible en un cierto espacio. Un ejemplo serían las líneas rectas, que son las geodésicas del plano.

            \item Enuncia los dos tipos de curvatura que hay y cuáles son las diferencias entre ambas. No necesitas mayor formalidad matemática que la que seguimos en clase. ¿A qué tipo de curvatura hace referencia el enunciado ``la gravedad está codificada en la curvatura del espacio-tiempo''?
            
            \inlinesol

            Existen dos tipos de curvatura: curvatura intrínseca y curvatura extrínseca.

            \begin{itemize}[label = \textbullet]
                \item La \textbf{curvatura intrínseca} es la curvatura que únicamente es percibida por los \emph{elementos} de ese espacio. 
                \item La \textbf{curvatura extrínseca} es la curvatura percibida desde un espacio de mayor dimensionalidad, fuera del espacio al que está asociada la curvatura.
            \end{itemize}

            ¿A qué tipo de curvatura hace referencia el enunciado ``la gravedad está codificada en la curvatura del espacio-tiempo''? La curvatura del espacio-tiempo es intrínseca, ya que por ahora no hay un espacio de mayor dimensionalidad que el espacio-tiempo que conocemos.

            \item Las ecuaciones de Einstein son
            
            \begin{equation}
                \tensor{R}{_{\mu\nu}} - \dfrac{1}{2}\tensor{g}{_{\mu\nu}}R = 8\pi\tensor{T}{_{\mu\nu}}. 
                \label{eq:EinsteinEqs}
            \end{equation}

            ¿En qué consiste encontrar una solución a éstas? Es decir, qué objetos necesitan darse como información y para qué objeto se resuelve.
        \end{enumerate}
    \end{exercise}
\end{document}