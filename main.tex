\documentclass[12pt]{article}
\usepackage{polyglossia}
\usepackage{fontspec}
\usepackage[export]{adjustbox}
\usepackage{amsfonts, amsmath, amssymb}
\usepackage{bm}
\usepackage{breqn}
\usepackage{derivative}
\usepackage{empheq}
\usepackage[inline]{enumitem}
\usepackage{fancyhdr}
\usepackage{geometry}
\usepackage{graphicx}
\usepackage{kantlipsum}
\usepackage{lastpage}
\usepackage{linebreaker} % line-breaker algorithm in LuaLaTeX
\usepackage{lualatex-math} % Fixes for mathematics
\usepackage{mathtools}
\usepackage{microtype}
\usepackage{mismath}
\usepackage{nicefrac}
\usepackage[hyperref]{ntheorem}
% \usepackage{phfqit} % Bra-ket notation
% \usepackage{physics}
\usepackage[final]{showlabels}
\usepackage{simples-matrices}
\usepackage{siunitx}
\usepackage{soul}
\usepackage{titling}
\usepackage{tabularray}
\usepackage{tasks}
% \usepackage{tensor}
\usepackage{xcolor}
\usepackage{xfrac}
\usepackage{xparse}
\usepackage{xspace}
\usepackage{hyperref}
% Polyglossia config
\setdefaultlanguage[variant=mexican]{spanish}
\setotherlanguage{english}
\usepackage{cleveref}

%%%%%%%%%%%%%%%%%%%%%%%%%%%%%%%%%%%%%%%%%%%%%%%%%%%%%%%%%%
%%%%%%%%%%%%%%%%%%%%%%%%%%%%%%%%%%%%%%%%%%%%%%%%%%%%%%%%%%
%%%%%%%%%%%%%%%%%%Configuraciones extras%%%%%%%%%%%%%%%%%%
\definecolor{base3}{RGB}{253, 246, 227}%
\definecolor{pinkwave}{RGB}{255, 0, 128}%
\definecolor{customBlue}{RGB}{11, 61, 98}%
\pagecolor{base3}
\graphicspath{{img/}}
\setlength{\parindent}{2em} % Sangría
\setlength{\parskip}{0.5em} % Espacio entre párrafos
\linespread{1.1} % line spacing
\setlength{\jot}{10pt} % Space between lines in multiline eqs
\crefname{equation}{ec.}{ecs.} % Equation's cross-reference name

% Line-breaker config
\linebreakersetup {
    maxtolerance = 90,
    maxemergencystretch = 1em,
    maxcycles = 4
}

%%%%%%%%%%%%%%%%%%Theorem environments%%%%%%%%%%%%%%%%%%
% Configuración de ambiente para problema
\theoremstyle{break}
\theoremheaderfont{\Large\normalfont\bfseries}
\theorembodyfont{\normalfont}
\theoremseparator{\bigskip} % Spacing between header and body
\theorempreskip{1.5em}
\theorempostskip{\topsep\bigskip}
\theorempostwork{
    \color{customBlue} \hrule width \hsize height 2pt \kern 1mm \hrule width \hsize
    }
\newtheorem{exercise}{Problema}
\counterwithin{equation}{exercise}

% Configuración de ambiente para solución
\theoremstyle{nonumberbreak}
\theoremheaderfont{\Large\normalfont\bfseries}
\theorembodyfont{\normalfont}
\theoremseparator{\medskip}
\theorempreskip{1em}
\theorempostskip{\topsep\medskip}
\newtheorem{solution}{Solución}

%%%%%%%%%%%%%%%%%%%%%%%%%%%%%%%%%%%%%%%%%%%%%%%%%%%%%%%%


% Configuración del paquete hyperref
\hypersetup{
    colorlinks = true,%
    linkcolor={[rgb]{0,0.2,0.6}},%
    citecolor={[rgb]{0,0.6,0.2}},%
    filecolor={[rgb]{0.8,0,0.8}},%
    urlcolor={[rgb]{0.8,0,0.8}},%
    runcolor={[rgb]{0.8,0,0.8}},% 
    menucolor={[rgb]{0,0.2,0.6}},%
    linkbordercolor={[rgb]{0,0.2,0.6}},%
    citebordercolor={[rgb]{0,0.6,0.2}},%
    filebordercolor={[rgb]{0.8,0,0.8}},%
    urlbordercolor={[rgb]{0.8,0,0.8}},%
    runbordercolor={[rgb]{0.8,0,0.8}},%
    menubordercolor={[rgb]{0,0.2,0.6}},% 
    pdftitle={Tarea 3},%
    pdfauthor={López Merino Marcos},%
    pdfsubject={Relatividad},%
    pdfkeywords={Facultad de Ciencias, UNAM, Relatividad},%
    unicode = true%
}

%%%%%%%%%%%%%%%%%%siunitx configuration%%%%%%%%%%%%%%%%%%
% Configuración del paquete siunitx
\sisetup{
	output-decimal-marker = {.}, 
	per-mode = symbol-or-fraction,
	separate-uncertainty = false,
	exponent-product = \cross,
    % inter-unit-product = \ensuremath{{}\vdot{}}
}

% Declaring new units
\DeclareSIUnit\kilogram{\kilo\gram}

%%%%%%%%%%%%%%%%%%%%%%%%%%%%%%%%%%%%%%%%%%%%%%%%%%%%%%%%

% Geometría del documento
\geometry{
    letterpaper,
    top = 0.6in,
    bottom = 0.8in,
    left = 0.6in,
    right = 0.6in,
    footskip = 38pt
}

%%%%%%%%%%%%%%%%%%Nuevos comandos%%%%%%%%%%%%%%%%%%
\newcommand*{\group}{8231}
\newcommand*{\classname}{Relatividad}
\newcommand*{\homeworknumber}{Tarea 3}
\newcommand*{\name}{Marcos López Merino}

% unit vector i
\newcommand*{\uveci}{{\bm{\hat{\textnormal{\bfseries\i}}}}}
% unit vector j
\newcommand*{\uvecj}{{\bm{\hat{\textnormal{\bfseries\j}}}}}
% unit vector
\DeclareRobustCommand{\uvec}[1]{{%
  \ifcsname uvec#1\endcsname
     \csname uvec#1\endcsname
   \else
    \bm{\hat{\mathbf{#1}}}%
   \fi
}}% 
\newcommand{\idest}{\emph{i.e.},\xspace} % id est
% Espacio vectorial, e.g., ℝ, ℂ, ℕ, etc.
\NewDocumentCommand{\vecspace}{m o}{%
  \IfValueTF{#2}{%
    \mathbb{#1}^{#2}%
  }{%
    \mathbb{#1}%
  }%
}

\newcommand*{\observer}{\mathcal{O}}
\newcommand*{\primeobserver}{\mathcal{O}^{\prime}}
\newcommand*{\biprimeobserver}{\mathcal{O}^{\prime\prime}}
\newcommand*{\triprimeobserver}{\mathcal{O}^{\prime\prime\prime}}
\newcommand*{\inlinesol}{\vspace*{10pt}\textbf{Solución}\vspace*{10pt}}
% \newcommand*{\change}[1]{\Delta{#1}\xspace}

\NewDocumentCommand{\change}{ s o m }{%
    \IfBooleanTF{#1}{%
        \Delta {#3^{\prime}}
    }{%
        \Delta {#3}
    }
}
\NewDocumentCommand{\interval}{ s o m }{%
    \IfBooleanTF{#1}{%
        (\Delta {#3^{\prime}})^{2}
    }{%
        (\Delta {#3})^{2}
    }
}
% \newcommand{\crefrangeconjunction}{--}
% \newcommand{\crefpairconjunction}{\xspace y\xspace}

% Defining a variant of Aboxed command from mathtools
\makeatletter
\newcommand*\Acolorboxed[2][pinkwave]{%
   \let\bgroup{\romannumeral-`}%
   \@Acolorboxed{#1}#2&&\ENDDNE
}
\def\@Acolorboxed#1#2&#3&#4\ENDDNE{%
  \ifnum0=`{}\fi
  \setbox\z@\hbox{$\displaystyle#2{}\m@th$\kern\fboxsep \kern\fboxrule}%
  \edef\@tempa{\kern\wd\z@ & \kern-\the\wd\z@ \fboxsep\the\fboxsep \fboxrule\the\fboxrule}%
  \@tempa
  \fcolorbox{#1}{base3}{\m@th$\displaystyle#2#3$}%
} 
\makeatother

%%%%%%%%%%%%%%%%%%Portada y configuración%%%%%%%%%%%%%%%%%%
% Configuración de portada
\setlength{\droptitle}{-60pt} % raise the title

% Portada
\title{
    \textbf{\homeworknumber}\\
    \normalsize\vspace{0.1in}\small{\textbf{Entrega}:~\today}
    \vspace{-1.5in}
}
\author{}
\date{}

%%%%%%%%%%%%%%%%%%Header and footer%%%%%%%%%%%%%%%%%%
\setlength{\headheight}{15.2pt}
\pagestyle{fancy}
\lhead{Grupo \group, Sem. 2023-1}
\chead{\classname}
\rhead{\name}
\lfoot{\includegraphics[scale = 0.2, valign = c]{LogoFCUNAMcolor.pdf}}
% \lfoot{\includegraphics[scale = 0.1, valign = c]{example-image}}
\cfoot{\homeworknumber}
\rfoot{Pág. \thepage \hspace{1pt} de \pageref{LastPage}}

\renewcommand{\headrulewidth}{0.5pt}
\renewcommand{\footrulewidth}{0.5pt}
%%%%%%%%%%%%%%%%%%%%%%%%%%%%%%%%%%%%%%%%%%%%%%%%%%%%%%%%%%
%%%%%%%%%%%%%%%%%%%%%%%%%%%%%%%%%%%%%%%%%%%%%%%%%%%%%%%%%%

\begin{document}
    \maketitle
    \thispagestyle{fancy}
    
    \begin{exercise}
        En clase estudiamos el fenómeno de dilatación temporal, imaginando la siguiente situación:\par

        Una mañana normal Flash patrulla las calles de Ciudad Central, cuando de pronto escucha una alarma a lo lejos: están robando un banco. Como el buen héroe que es, se pone en marcha de inmediato para evitarlo. Para llegar a tiempo simplemente necesita correr muy rápido, ¿no es cierto? Sin embargo, no se da cuenta que la velocidad a la que está corriendo está peligrosamente cerca a la velocidad de la luz, tanto que empieza a meterse con la fibra del espacio-tiempo (aquí imaginen efectos visuales típicos del espacio-tiempo, luces y colores psicodélicos). Cuando llega al banco, aunque según se reloj solo transcurrieron 30 segundos, se da cuenta de que no logró detener el asalto ¡Pues ocurrió hace 30 años según el reloj de la pared del banco!\par

        La expresión (con el factor de \(c\) restituido) a la que llegamos en clase es

        \begin{equation*}
            \change{t} = \dfrac{\change*{t}}{\sqrt{1 - \frac{v^{2}}{v^{2}}}},
        \end{equation*}

        donde el sistema \(\observer\) corresponde a un observador en Ciudad Central, \(\primeobserver\) corresponde a Flash.

        \begin{enumerate}[label = \alph*)]
            \item Calcula la velocidad \(v\) con la que Flash corrió hacia el banco según \(\observer\) (mucho cuidado con los decimales).
            \item Si Flash hubiera viaja a \(v = 0.9c\), ¿cuánto tiempo habría tardado en llegar al banco según \(\observer\)? Supón que para Flash aún pasaron solo 30 segundos. La respuesta a este problema debe darte una idea de las velocidades necesarias para que los efectos relativistas sean perceptibles.
        \end{enumerate}
    \end{exercise}

    \begin{exercise}
        En este problema estudiarás una forma para deducir las transformaciones de Lorentz alternativa a la que dimos en clase. Considera un espacio-tiempo 1+1-dimensional y dos observadores en éste, \(\observer\) y \(\primeobserver\), tales que el segundo se mueve con velocidad \(v\) respecto al primero y sus orígenes coinciden. Nota primero que podemos escribir al intervalo entre dos eventos cualesquiera en notación matricial
        
        \begin{equation}
            \interval{s} =
                        \matrice[2]{\change{t}, \change{x}}
                        \matrice[2]{-1, 0, 0, 1}
                        \matrice[1]{\change{t}, \change{x}}.
            \label{eq:STInterval}
        \end{equation}

        La matriz que aparece en esa expresión

        \begin{equation*}
            \eta =
                \matrice[2]{-1, 0, 0, 1}
        \end{equation*}

        es muy importante. De hecho, lo que harás es definir a las transformaciones de Lorentz haciendo referencia a esta matriz y no al intervalo en sí.

        \begin{enumerate}[label = \alph*)]
            \item Escribe la transformación lineal entre los sistemas \(\observer\) y \(\primeobserver\) en forma matricial. Vas a deducir las entradas de dicha matriz \(\Lambda\) en este problema, así que por ahora usa las letras que gustes como incógnitas. 
            \item Escribe la relación entre \((\change{t}, \change{x})\) y \((\change*{t}, \change*{x})\) de forma matricial usando a).
            \item Usa (\ref{eq:STInterval}) para escribir una invariancia del intervalo \(\interval{s} = \interval*{s}\) en forma matricial. Luego usa b) para escribir todo en términos de \((\change{t}, \change{x})\).
            \item Dado que tu expresión de c) debe ser válida para cualquier \((\change{t}, \change{x})\), ¿qué relación deben cumplir \(\eta\) y \(\Lambda\)? Se dice entonces que \(\Lambda\) preserva a la matriz \(\eta\).
            \item Resuelve la expresión que encontraste en d), es decir, resuelve el sistema de ecuaciones para determinar todas las entradas de la matriz de transformación \(\Lambda\). Tu resultado debe quedar expresado en términos de un solo parámetro libre \(\rho\). Notarás que hay dos tipos de soluciones, las que queremos en este problema es la que tiene determinante positivo. \emph{Hint}: ¿Qué relación cumplen \(\sinh(\rho)\) y \(\cos(\rho)\) para cualquier \(\rho\)?
            \item La matriz \(\Lambda\) a la que llegaste luce algo diferente a la matriz
            
            \begin{equation*}
                \Lambda =
                        \matrice[2]{\frac{1}{\sqrt{1 - v^{2}}},
                                    -\frac{v}{\sqrt{1 - v^{2}}},
                                    -\frac{v}{\sqrt{1 - v^{2}}},
                                    \frac{1}{\sqrt{1 - v^{2}}}}
            \end{equation*}

            que obtuvimos en clase. Sin embargo, son la misma matriz. ¿Qué relación debe haber entre \(\rho\) y \(v\) para que esto ocurra? A \(\rho\) se le conoce con el nombre de \textbf{rapidity}.
        \end{enumerate}
    \end{exercise}

    \begin{exercise}
        Sean \(\observer\) y \(\primeobserver\) dos sistemas inerciales en un espacio-tiempo 2+1-dimensional, tales que \(\primeobserver\) se mueve con velocidad \(\vect{v} = \cos\theta\uvec{x} + \sin\theta\uvec{y}\) respecto a \(\observer\). Los dos sistemas comparte el mismo origen y son tales que sus ejes espaciales están alineados, es decir, no hay una rotación ni una traslación de por medio entre los sistemas \(\observer\) y \(\primeobserver\), la única diferencia entre ambos es su estado de movimiento. En este problema la transformación de Lorentz entre las coordenadas de ambos sistemas. Para esto:

        \begin{enumerate}
            \item Considera primero un sistema intermedio \(\biprimeobserver\), tal que este se encuentra en reposo respecto a \(\observer\), comparte el mismo origen coordenado, pero la velocidad de \(\primeobserver\) respecto a \(\biprimeobserver\) es \(\vect{v} = v\uvec{x}^{\prime\prime}\). Escribe la transformación de coordenadas para pasar de \(\observer\) a \(\biprimeobserver\) (las tres coordenadas, el tiempo incluido).
            \item Usa un boost de Lorentz en la dirección \(x^{\prime\prime}\) con la velocidad \(v\) para de \(\biprimeobserver\) a otro sistema intermedio \(\triprimeobserver\). Nota que este nuevo sistema está en reposo respecto a \(\primeobserver\), pero sus ejes espaciales no coinciden debido a la rotación que aplicaste en a) (hacer un dibujo puee ayudarte a visualizar esto).
            \item Usa ahora una rotación para pasar del sistema \(\triprimeobserver\) a \(\primeobserver\).
            \item Finalmente, usa tus resultados de todos los incisos anteriores para escribir explícitamente la relación entre las coordenadas de \(\observer\) con las coordenadas de \(\primeobserver\).
        \end{enumerate}
    \end{exercise}
\end{document}